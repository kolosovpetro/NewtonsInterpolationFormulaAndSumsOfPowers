\documentclass[12pt,letterpaper,oneside,reqno]{amsart}
\usepackage{amsfonts}
\usepackage{amsmath}
\usepackage{amssymb}
\usepackage{amsthm}
\usepackage{float}
\usepackage{mathrsfs}
\usepackage{colonequals}
\usepackage[font=small,labelfont=bf]{caption}
\usepackage[unicode,pdfpagelabels,hyperindex,colorlinks=true,linkcolor=red,urlcolor=blue,citecolor=red]{hyperref}
\usepackage{graphicx}
\emergencystretch=1em
\usepackage{array}
\usepackage{enumitem}
\usepackage{etoolbox}
\usepackage{physics}
\usepackage{booktabs}
\usepackage{url}
\usepackage{mdframed}

% margins and layout
\linespread{1.7}
\usepackage[left=1in,right=1in,bottom=1in,top=1in]{geometry}
\apptocmd{\sloppy}{\hbadness 10000\relax}{}{}
\raggedbottom

%\newcommand \coeffA [3][A] {{\mathbf{#1}} \sb{#2,#3}}
%\newcommand \polynomialP [4][P]{{\mathbf{#1}}\sp{#2} \sb{#3}(#4)}
\newcommand \bernoulli [2][B] {{#1}\sb{#2}}

%\newcommand \anglePower [2]{\langle #1 \rangle \sp{#2}}
%\newcommand \curvePower [2]{\{#1\}\sp{#2}}

% central factorials and related symbols
\newcommand \centralFactorial [2] {#1^{[#2]}}
\newcommand \fallingFactorial [2] {\left(#1 \right)^{\underline{#2}}}
\newcommand{\stirlingii}{\genfrac{\{}{\}}{0pt}{}}
\newcommand{\eulerianNumber}{\genfrac{\langle}{\rangle}{0pt}{}}

\newcommand{\KnuthRFoldSum}[3]{\Sigma^{#1}\,{#2}^{#3}}

% for llceil coeffcient
%\newcommand{\nobarfrac}{\genfrac{}{}{0pt}{}}
%\def\llceil{\left\lceil\kern-3.5pt\left\lceil}
%\def\rrfloor{\right\rfloor\kern-3.5pt\right\rfloor}
%\newcommand \llceilCoefficient [3] {\llceil \nobarfrac{#1}{#2} \rrfloor_{#3}}

% ~~~ Rascal numbers ~~~
%\newcommand \rascalNumber [3] {\binom{#1}{#2}_{#3}}
%\newcommand \north[0] {\mathbf{North}}
%\newcommand \south[0] {\mathbf{South}}
%\newcommand \west[0] {\mathbf{West}}
%\newcommand \east[0] {\mathbf{East}}

% ~~~~ 1-q pascal notation~~~~
%\newcommand{\genstirlingI}[3]{%
%    \genfrac{[}{]}{0pt}{#1}{#2}{#3}%
%}
%\newcommand{\genstirlingII}[3]{%
%    \genfrac{\{}{\}}{0pt}{#1}{#2}{#3}%
%}
%\newcommand{\oneQBinomial}[3]{\genstirlingI{}{#1}{#2}^{#3}}


\newtheorem{theorem}{Theorem}[section]
\newtheorem{corollary}[theorem]{Corollary}
\newtheorem{proposition}[theorem]{Proposition}
\newtheorem{observation}[theorem]{Observation}
\newtheorem{lemma}[theorem]{Lemma}
\newtheorem{claim}[theorem]{Claim}
\newtheorem{example}[theorem]{Example}
\newtheorem{conjecture}[theorem]{Conjecture}
\newtheorem{definition}[theorem]{Definition}
\newtheorem{question}[theorem]{Question}
\newtheorem{remark}[theorem]{Remark}
\newtheorem{assumption}[theorem]{Assumption}

%\numberwithin{equation}{section}

\title[Newton's interpolation formula and sums of powers]
{Newton's interpolation formula and sums of powers}
\author[Petro Kolosov]{Petro Kolosov}
\date{\today}

% metadata
%\email{kolosovp94@gmail.com}
\address{DevOps Engineer}
\urladdr{https://kolosovpetro.github.io}
\subjclass[2010]{05A19, 05A10, 41A15, 11B83}
\keywords{Sums of powers,
    Newton's interpolation formula,
    Finite differences,
    Binomial coefficients,
    Faulhaber's formula,
    Bernoulli numbers,
    Bernoulli polynomials,
    Interpolation,
    Combinatorics,
    Central factorial numbers,
    Stirling numbers,
    Eulerian numbers,
    Worpitzky identity,
    OEIS}
\hypersetup{
    pdftitle={Newton's interpolation formula and sums of powers},
    pdfproducer={LaTeX},
    pdfcreator={pdflatex},
    pdfauthor={Petro Kolosov},
    pdfsubject={Newton's interpolation formula and sums of powers},
    pdfkeywords={Finite differences,
    Interpolation,
    Sums of powers,
    Stirling numbers,
    Eulerian numbers,
    Power sums,
    Binomial theorem,
    Power function,
    Binomial coefficients,
    Bernoulli numbers,
    Bernoulli polynomials,
    Worpitzky identity,
    Pascal's triangle,
    Faulhaber's formula,
    OEIS,
    Combinatorics,
    Central factorial numbers}
}

\begin{document}

    \maketitle

    \begin{abstract}
        In this manuscript we derive the formulas for multifold sums of powers by utilizing
Newton's interpolation formula.
Furthermore, this manuscript provides the formulas for multifold sums of powers in terms of
Stirling numbers of the second kind, and Eulerian numbers.

    \end{abstract}

    \tableofcontents


    \section{Introduction and main results}
    \label{sec:introduction}
    \begin{proposition}
    \label{proposition:newton-series-in-arbitrary-point-a}
    (Newton's series around arbitrary point~\cite[Lemma V]{newton1850newton}.)
        \begin{align*}
            f(x) = \sum_{j=0}^{\infty} \binom{x-a}{j} \Delta^{j} f(a)
        \end{align*}
\end{proposition}

\begin{example}[Newton series for cubes monomial]
    \label{example:newton-series-for-cubes-monomial}
    \begin{align*}
        n^3 &= 0 \binom{n}{0} + 1 \binom{n}{1} + 6 \binom{n}{2} + 6 \binom{n}{3} \\
        n^3 &= 1 \binom{n-1}{0} + 7 \binom{n-1}{1} + 12 \binom{n-1}{2} + 6 \binom{n-1}{3} \\
        n^3 &= 8 \binom{n-2}{0} + 19 \binom{n-2}{1} + 18 \binom{n-2}{2} + 6 \binom{n-2}{3}
    \end{align*}
    \begin{mdframed}
        In general,
        \begin{align*}
            n^3 &= \Delta^0 t^3 \binom{n-t}{0} + \Delta^1 t^3 \binom{n-t}{1} + \Delta^2 t^3 \binom{n-t}{2} + \Delta^3 t^3 \binom{n-t}{3}
        \end{align*}
    \end{mdframed}
\end{example}

\begin{corollary}[Newton series for binomial reversed]
    \label{corollary:newton-series-for-binomial-reversed}
    \begin{align*}
    (n+t)
        ^m &= \sum_{k=0}^{m} \binom{n}{k} \Delta^{k} t^m
    \end{align*}
\end{corollary}

\begin{proposition}[Newton series for monomial reversed]
    \label{prop:newton-series-for-monomial-reversed}
    \begin{align*}
        n^m = \sum_{k=0}^{m} \binom{n-t}{k} \Delta^{k} t^m
    \end{align*}
    \begin{proof}
        By setting $n \rightarrow n-t$ into~\eqref{corollary:newton-series-for-binomial-reversed}.
    \end{proof}
\end{proposition}

\begin{definition}[Multifold sum of powers recurrence]
    \begin{align*}
        \KnuthRFoldSum{0}{n}{m}   &= n^m \\
        \KnuthRFoldSum{1}{n}{m}   &= \KnuthRFoldSum{0}{1}{m} + \KnuthRFoldSum{0}{2}{m} + \cdots + \KnuthRFoldSum{0}{n}{m} \\
        \KnuthRFoldSum{r+1}{n}{m} &= \KnuthRFoldSum{r}{1}{m} + \KnuthRFoldSum{r}{2}{m} + \cdots + \KnuthRFoldSum{r}{n}{m}
    \end{align*}
\end{definition}
Thus, for arbitrary integer $t$
\begin{align*}
    \KnuthRFoldSum{1}{n}{m}
    = \sum_{k=1}^{n} \sum_{j=0}^{m} \binom{-t+k}{j} \Delta^{j} t^m
    = \sum_{j=0}^{m} \Delta^{j} t^m  \sum_{k=1}^{n} \binom{-t+k}{j}
\end{align*}
\begin{proposition}[Segmented Hockey stick identity]
    \label{prop:segmented-hockey-stick-identity}
    For integers $n,t$ and $j$
    \begin{align*}
        \sum_{k=0}^{n} \binom{-t+k}{j} &= (-1)^j \binom{j+t}{j+1} +  \binom{n-t+1}{j+1}
    \end{align*}
\end{proposition}
Therefore,
\begin{proposition}[Ordinary sums of powers via Newton's series]
    \label{prop:ordinary-sums-of-powers-via-newtons-series}
    For non-negative integers $n,m$ and arbitrary integer $t$
    \begin{align*}
        \KnuthRFoldSum{1}{n}{m} = \sum_{j=0}^{m} \Delta^{j} t^m \left[ (-1)^j \binom{j+t-1}{j+1} +  \binom{n-t+1}{j+1} \right]
    \end{align*}
    \begin{proof}
        Ordinary sum of powers is given by $\KnuthRFoldSum{1}{n}{m} = \sum_{j=0}^{m} \Delta^{j} t^m  \sum_{k=1}^{n} \binom{-t+k}{j}$,
        where $\sum_{k=1}^{n} \binom{-t+k}{j} =  (-1)^{j} \binom{j+t-1}{j+1} + \binom{n-t+1}{j+1}$
        by means of segmented hockey stick identity~\eqref{prop:segmented-hockey-stick-identity}.
    \end{proof}
\end{proposition}
The special cases for $t=0$ and $t=1$ are widely known and appear in literature quite frequently.
For $t=0$ and $m=3$ we have the famous identity
\begin{align*}
    \KnuthRFoldSum{1}{n}{3} = 0 \binom{n+1}{1} + 1 \binom{n+1}{2} + 6 \binom{n+1}{3} + 6 \binom{n+1}{4}
\end{align*}
which was discussed in~\cite[p. 190]{graham1994concrete} and in~\cite{pfaff2007deriving}.
The special cases for $t=1$ and $m=2,3,4,5$ were discussed in~\cite{cereceda2022sums}.
For instance,
\begin{align*}
    \KnuthRFoldSum{1}{n}{3} &= 1 \binom{n}{1} + 7 \binom{n}{2} + 12 \binom{n}{3} + 6 \binom{n}{4} \\
    \KnuthRFoldSum{1}{n}{4} &= 1 \binom{n}{1} +15 \binom{n}{2} +50 \binom{n}{3} +60 \binom{n}{4} +24 \binom{n}{5}
\end{align*}
The coefficients $1,7,12, \dots$ are given by the sequence [ID] in the OEIS~\cite{sloane2003line}.
Interestingly enough that the paper~\cite{cereceda2022sums} gives the formula for sums of powers
\begin{align*}
    \KnuthRFoldSum{1}{n}{k} = \sum_{j=0}^{k} j! \left[ \binom{n+1-r}{j+1} + (-1)^j \binom{r+j-1}{j+1} \right] \stirlingii{k}{j}_{r}
\end{align*}
where $\stirlingii{k}{j}_{r}$ are generalized Stirling numbers of the second kind.
The formula above is identical to the proposition~\eqref{prop:ordinary-sums-of-powers-via-newtons-series},
which yields that finite differences can be expressed in terms of generalized Stirling numbers of the second kind,
that is $\Delta^{j} t^m = j! \stirlingii{m}{j}_{t}$.

By considering the special cases of the theorem~\eqref{prop:ordinary-sums-of-powers-via-newtons-series} for $t=4$,
we observe rather unexpected formulas for sums of powers, that are
\begin{align*}
    \KnuthRFoldSum{1}{n}{0} &= 1  \left( \binom{n-3}{1} + \binom{3}{1}  \right) \\
    \KnuthRFoldSum{1}{n}{1} &= 4  \left( \binom{n-3}{1} + \binom{3}{1}  \right)  + 1  \left( \binom{n-3}{2} - \binom{4}{2}  \right) \\
    \KnuthRFoldSum{1}{n}{2} &= 16 \left( \binom{n-3}{1} + \binom{3}{1}  \right)  + 9  \left( \binom{n-3}{2} - \binom{4}{2}  \right) + 2  \left( \binom{n-2}{3} + \binom{5}{3}  \right) \\
    \KnuthRFoldSum{1}{n}{3} &= 64 \left( \binom{n-3}{1} + \binom{3}{1}  \right)  + 61 \left( \binom{n-3}{2} - \binom{4}{2}  \right) + 30 \left( \binom{n-3}{3} + \binom{5}{3}  \right) \\ &+ 6 \left( \binom{n-3}{4} - \binom{6}{4}  \right)
\end{align*}
The coefficients $1,4,1,16,9,\dots$ are given by the sequence [ID] in the OEIS~\cite{sloane2003line}.
To obtain the formula for double sum of powers, we simply apply summation operator over the ordinary sum again, thus
\begin{align*}
    \KnuthRFoldSum{2}{n}{m} = \sum_{j=0}^{m} \Delta^{j} t^{m} \left[ (-1)^{j} \sum_{k=1}^{n} \binom{j+t-1}{j+1} + \sum_{k=1}^{n} \binom{k-t+1}{j+1} \right]
\end{align*}
which yields
\begin{align*}
    \KnuthRFoldSum{2}{n}{m} = \sum_{j=0}^{m} \Delta^{j} t^{m} \left[ (-1)^{j} \binom{j+t-1}{j+1} n + \sum_{k=1}^{n} \binom{k-t+1}{j+1} \right]
\end{align*}
Thus,
\begin{proposition}[Double sums of powers via Newton's series]
    \begin{align*}
        \KnuthRFoldSum{2}{n}{m} = \sum_{j=0}^{m} \Delta^{j} t^{m} \left[ (-1)^{j} \binom{j+t-1}{j+1} n + (-1)^{j+1} \binom{j+t-1}{j+2} n^0 + \binom{n-t+2}{j+2} \right]
    \end{align*}
    \begin{proof}
        We have $\KnuthRFoldSum{2}{n}{m} = \sum_{j=0}^{m} \Delta^{j} t^{m} \left[ (-1)^{j} \binom{j+t-1}{j+1} n + \sum_{k=1}^{n} \binom{k-t+1}{j+1} \right]$,
        where $\sum_{k=1}^{n} \binom{k-t+1}{j+1} = (-1)^{j+1} \binom{j+t-1}{j+2} n^0 + \binom{n-t+2}{j+2}$ by means of segmented hockey stick
        identity~\eqref{prop:segmented-hockey-stick-identity}.
    \end{proof}
\end{proposition}
For example, given $t=5$, the double sums of powers are
\begin{align*}
    \KnuthRFoldSum{2}{n}{0} &= 1 \left( \binom{n-3}{2} + \binom{4}{1} n - \binom{4}{2} \right) \\
    \KnuthRFoldSum{2}{n}{1} &= 5 \left( \binom{n-3}{2} + \binom{4}{1} n - \binom{4}{2} \right) + 1 \left( \binom{n-3}{3} - \binom{5}{2} n + \binom{5}{3} \right) \\
    \KnuthRFoldSum{2}{n}{2} &= 25 \left( \binom{n-3}{2} + \binom{4}{1} n - \binom{4}{2} \right) + 11 \left( \binom{n-3}{3} - \binom{5}{2} n + \binom{5}{3} \right) \\
    &+ 2 \left( \binom{n-3}{4} + \binom{6}{3} n - \binom{6}{4} \right) \\
    \KnuthRFoldSum{2}{n}{3} &= 125 \left( \binom{n-3}{2} + \binom{4}{1} n - \binom{4}{2} \right) + 91 \left( \binom{n-3}{3} - \binom{5}{2} n + \binom{5}{3} \right) \\
    &+ 36 \left( \binom{n-3}{4} + \binom{6}{3} n - \binom{6}{4} \right) + 6 \left( \binom{n-3}{5} - \binom{7}{4} n + \binom{7}{5} \right)
\end{align*}
Similarly, we obtain the formula for the triple sums of powers



    \section{Backward difference form}
    \label{sec:backward-difference-form}
    The formula for multifold sums of powers via Newton's series~\eqref{theorem:multifold-sums-of-powers-via-newtons-series}
can be altered to be in terms of backward differences easily, because
\begin{align*}
    \nabla^{j} (t+1)^{m} = \Delta^{j} t^{m}
\end{align*}
Thus,
\begin{proposition}[Multifold sums of powers via backward difference]
    \label{prop:multifold-sums-of-powers-via-backward-difference}
    For non-negative integers $r,n,m$ and an arbitrary integer $t$
    \begin{align*}
        \KnuthRFoldSum{r}{n}{m} = \sum_{j=0}^{m} \nabla^{j} (t+1)^{m} \left[ \left( \sum_{s=1}^{r} (-1)^{j+s-1} \binom{j+t-1}{j+s} \KnuthRFoldSum{r-s}{n}{0} \right) + \binom{n-t+r}{j+r} \right]
    \end{align*}
    \begin{proof}
        By multifold sums of powers via Newton's series~\eqref{theorem:multifold-sums-of-powers-via-newtons-series} and
        the identity $\nabla^{j} (t+1)^{m} = \Delta^{j} t^{m}$.
    \end{proof}
\end{proposition}



    \section{Future research}
    \label{sec:future-research}
    In this manuscript we focus on the idea to combine the Newton's interpolation formula and Hockey-stick identity
for binomial coefficients to express the sums of powers seamlessly.

This particular idea is great, however it can be generalized even further, so that the main aim is to
utilize an interpolation formula for power $n^m$ in terms of \textit{abstract difference operator} $D(n^m)$ and binomial coefficients $\binom{f(n)}{k}$
such that $n$ indicates the variable of power function.
The difference operator can be arbitrary, for example: forward, backward, central differences etc.
For example, the abstract interpolation formula is
\begin{align*}
    n^m = \sum_{k} \binom{f(n)}{k} D(n^m, k)
\end{align*}
Thus, the formula of sums of powers involves the abstract difference operator $D$ in some point $k$ and hockey stick
identity over the binomial coefficients $\binom{n}{k}$
\begin{align*}
    \KnuthRFoldSum{1}{n}{m} = \sum_{k} D(n^m, k) \sum_{j \leq n} \binom{f(j)}{k}
\end{align*}
Similarly, for multifold sums of powers
\begin{align*}
    \KnuthRFoldSum{r}{n}{m} = \sum_{k} D(n^m, k) \binom{f(n+r)}{k}
\end{align*}
Many of interpolation approaches involve rising factorials $x_{(n)}$, falling factorials $(x)_n$ or usual factorials $n!$,
and thus can be expressed
in terms of binomial coefficients, because
\begin{align*}
    \frac{(x)_n}{n!} = \binom{x}{n}; \quad \frac{x_{(n)}}{n!} = \binom{x+n-1}{n}.
\end{align*}

In particular, Donald Knuth provides the formula multifold sums of odd powers~\cite{knuth1993johann}
such that based on the operator of central finite differences of power evaluated in zero, that is
\begin{proposition}[Multifold sums of odd powers]
    \label{prop:knuth-sums-of-odd-powers}
    \begin{align*}
        \KnuthRFoldSum{r}{n}{2m-1}
        &= \sum_{k=1}^{m} (2k-1)! T(2m, 2k) \binom{n+k-1+r}{2k-1+r} \\
        &= \sum_{k=1}^{m} \binom{n+k-1+r}{2k-1+r} \frac{1}{2k} \delta^{2k} 0^{2m}
    \end{align*}
\end{proposition}
where $T(n,k)$ are central factorial numbers of the second
kind, see~\cite[section 58]{steffensen1927interpolation} and~\cite[formula (10a)]{carlitz1963divided},
such that
\begin{align*}
    T(n, k) = \frac{1}{k!} \delta^k 0^n = \frac{1}{k!} \sum_{j=0}^{k} (-1)^j \binom{k}{j} \left( \frac{k}{2} - j \right)^n
\end{align*}
In general, the central factorial numbers of the second kind $T(n,k)$ were defined by Riordan in his
fundamental work \textit{Combinatorial identities}~\cite[ch. 6.5, formula (24)]{riordan1968combinatorial},
via the polynomial identity
\begin{lemma} [Riordan power identity]
    \begin{align*}
        n^m = \sum_{k=1}^{m} T(m,k) \centralFactorial{n}{k}
    \end{align*}
\end{lemma}
where $\centralFactorial{n}{k}$ are central factorials
$\centralFactorial{n}{k} = n \prod_{j=0}^{k-1} \left( n + \frac{k}{2} -j \right)$.
The sequence
\href{https://oeis.org/A008957}{\texttt{A008957}}
in the OEIS~\cite{sloane2003line} provides non-zero central factorial numbers of the second kind $T(2n,2k)$.

The Knuth's formula~\eqref{prop:knuth-sums-of-odd-powers} utilizes the operator
of central finite differences of power evaluated in zero,
it is worth to research the existence of the sums of odd powers involving the central differences
evaluated in arbitrary integer point $t$,
similar to multifold sums of powers via Newton's series~\eqref{theorem:multifold-sums-of-powers-via-newtons-series}.



    \section{Proof of Segmented hockey stick identity}
    \label{sec:proof-of-segmented-hockey-stick-identity}
    \input{sections/proof_of_segmented_hockey_stick_identity}


%    \section{Conclusions}\label{sec:conclusions}
%    In this manuscript we have discussed the formulas for multifold sums of powers by utilizing
Newton's interpolation formula.
Furthermore, this manuscript provides the formulas for multifold sums of powers in terms of
Stirling numbers of the second kind, and Eulerian numbers.
In addition, in~\eqref{sec:future-research} we discussed the future research directions, that may lead to a complete framework
for sums of powers, by means of combining interpolation approaches, binomial coefficients and variations of
hockey-stick identity.
The most important results of this manuscript is validated using Mathematica programs, see~\eqref{sec:mathematica-programs}.



    \section{Acknowledgements}\label{sec:acknowledgements}
    The author is grateful to Markus Scheuer for his valuable contribution of the list of
references~\cite{RiordanCentralFactorialMSE2020} about the fact that central factorial numbers of the second kind
arise from central differences of nothing.


    \bibliographystyle{unsrt}
    \bibliography{NewtonsInterpolationFormulaAndSumsOfPowers}

    \input{metadata/license}

\end{document}
