\begin{proposition}
    \label{proposition:newton-series-in-arbitrary-point-a}
    (Newton's series around arbitrary point~\cite[Lemma V]{newton1850newton}.)
        \begin{align*}
            f(x) = \sum_{j=0}^{\infty} \binom{x-a}{j} \Delta^{j} f(a)
        \end{align*}
\end{proposition}

\begin{example}[Newton series for cubes monomial]
    \label{example:newton-series-for-cubes-monomial}
    \begin{align*}
        n^3 &= 0 \binom{n}{0} + 1 \binom{n}{1} + 6 \binom{n}{2} + 6 \binom{n}{3} \\
        n^3 &= 1 \binom{n-1}{0} + 7 \binom{n-1}{1} + 12 \binom{n-1}{2} + 6 \binom{n-1}{3} \\
        n^3 &= 8 \binom{n-2}{0} + 19 \binom{n-2}{1} + 18 \binom{n-2}{2} + 6 \binom{n-2}{3}
    \end{align*}
    \begin{mdframed}
        In general,
        \begin{align*}
            n^3 &= \Delta^0 t^3 \binom{n-t}{0} + \Delta^1 t^3 \binom{n-t}{1} + \Delta^2 t^3 \binom{n-t}{2} + \Delta^3 t^3 \binom{n-t}{3}
        \end{align*}
    \end{mdframed}
\end{example}

\begin{corollary}[Newton series for binomial reversed]
    \label{corollary:newton-series-for-binomial-reversed}
    \begin{align*}
    (n+t)
        ^m &= \sum_{k=0}^{m} \binom{n}{k} \Delta^{k} t^m
    \end{align*}
\end{corollary}

\begin{proposition}[Newton series for monomial reversed]
    \label{prop:newton-series-for-monomial-reversed}
    \begin{align*}
        n^m = \sum_{k=0}^{m} \binom{n-t}{k} \Delta^{k} t^m
    \end{align*}
    \begin{proof}
        By setting $n \rightarrow n-t$ into~\eqref{corollary:newton-series-for-binomial-reversed}.
    \end{proof}
\end{proposition}

\begin{definition}[Multifold sum of powers recurrence]
    \begin{align*}
        \KnuthRFoldSum{0}{n}{m}   &= n^m \\
        \KnuthRFoldSum{1}{n}{m}   &= \KnuthRFoldSum{0}{1}{m} + \KnuthRFoldSum{0}{2}{m} + \cdots + \KnuthRFoldSum{0}{n}{m} \\
        \KnuthRFoldSum{r+1}{n}{m} &= \KnuthRFoldSum{r}{1}{m} + \KnuthRFoldSum{r}{2}{m} + \cdots + \KnuthRFoldSum{r}{n}{m}
    \end{align*}
\end{definition}
Thus, for arbitrary integer $t$
\begin{align*}
    \KnuthRFoldSum{1}{n}{m}
    = \sum_{k=1}^{n} \sum_{j=0}^{m} \binom{-t+k}{j} \Delta^{j} t^m
    = \sum_{j=0}^{m} \Delta^{j} t^m  \sum_{k=1}^{n} \binom{-t+k}{j}
\end{align*}
\begin{proposition}[Segmented Hockey stick identity]
    \label{prop:segmented-hockey-stick-identity}
    For integers $n,t$ and $j$
    \begin{align*}
        \sum_{k=0}^{n} \binom{-t+k}{j} &= (-1)^j \binom{j+t}{j+1} +  \binom{n-t+1}{j+1}
    \end{align*}
\end{proposition}
Therefore,
\begin{proposition}[Ordinary sums of powers via Newton's series]
    \label{prop:ordinary-sums-of-powers-via-newtons-series}
    For non-negative integers $n,m$ and arbitrary integer $t$
    \begin{align*}
        \KnuthRFoldSum{1}{n}{m} = \sum_{j=0}^{m} \Delta^{j} t^m \left[ (-1)^j \binom{j+t-1}{j+1} +  \binom{n-t+1}{j+1} \right]
    \end{align*}
    \begin{proof}
        Ordinary sum of powers is given by $\KnuthRFoldSum{1}{n}{m} = \sum_{j=0}^{m} \Delta^{j} t^m  \sum_{k=1}^{n} \binom{-t+k}{j}$,
        where $\sum_{k=1}^{n} \binom{-t+k}{j} =  (-1)^{j} \binom{j+t-1}{j+1} + \binom{n-t+1}{j+1}$
        by means of segmented hockey stick identity~\eqref{prop:segmented-hockey-stick-identity}.
    \end{proof}
\end{proposition}
The special cases for $t=0$ and $t=1$ are widely known and appear in literature quite frequently.
For $t=0$ and $m=3$ we have the famous identity
\begin{align*}
    \KnuthRFoldSum{1}{n}{3} = 0 \binom{n+1}{1} + 1 \binom{n+1}{2} + 6 \binom{n+1}{3} + 6 \binom{n+1}{4}
\end{align*}
which was discussed in~\cite[p. 190]{graham1994concrete} and in~\cite{pfaff2007deriving}.
The special cases for $t=1$ and $m=2,3,4,5$ were discussed in~\cite{cereceda2022sums}.
For instance,
\begin{align*}
    \KnuthRFoldSum{1}{n}{3} &= 1 \binom{n}{1} + 7 \binom{n}{2} + 12 \binom{n}{3} + 6 \binom{n}{4} \\
    \KnuthRFoldSum{1}{n}{4} &= 1 \binom{n}{1} +15 \binom{n}{2} +50 \binom{n}{3} +60 \binom{n}{4} +24 \binom{n}{5}
\end{align*}
The coefficients $1,7,12, \dots$ are given by the sequence [ID] in the OEIS~\cite{sloane2003line}.
Interestingly enough that the paper~\cite{cereceda2022sums} gives the formula for sums of powers
\begin{align*}
    \KnuthRFoldSum{1}{n}{k} = \sum_{j=0}^{k} j! \left[ \binom{n+1-r}{j+1} + (-1)^j \binom{r+j-1}{j+1} \right] \stirlingii{k}{j}_{r}
\end{align*}
where $\stirlingii{k}{j}_{r}$ are generalized Stirling numbers of the second kind.
The formula above is identical to the proposition~\eqref{prop:ordinary-sums-of-powers-via-newtons-series},
which yields that finite differences can be expressed in terms of generalized Stirling numbers of the second kind,
that is $\Delta^{j} t^m = j! \stirlingii{m}{j}_{t}$.

By considering the special cases of the theorem~\eqref{prop:ordinary-sums-of-powers-via-newtons-series} for $t=4$,
we observe rather unexpected formulas for sums of powers, that are
\begin{align*}
    \KnuthRFoldSum{1}{n}{0} &= 1  \left( \binom{n-3}{1} + \binom{3}{1}  \right) \\
    \KnuthRFoldSum{1}{n}{1} &= 4  \left( \binom{n-3}{1} + \binom{3}{1}  \right)  + 1  \left( \binom{n-3}{2} - \binom{4}{2}  \right) \\
    \KnuthRFoldSum{1}{n}{2} &= 16 \left( \binom{n-3}{1} + \binom{3}{1}  \right)  + 9  \left( \binom{n-3}{2} - \binom{4}{2}  \right) + 2  \left( \binom{n-2}{3} + \binom{5}{3}  \right) \\
    \KnuthRFoldSum{1}{n}{3} &= 64 \left( \binom{n-3}{1} + \binom{3}{1}  \right)  + 61 \left( \binom{n-3}{2} - \binom{4}{2}  \right) + 30 \left( \binom{n-3}{3} + \binom{5}{3}  \right) \\ &+ 6 \left( \binom{n-3}{4} - \binom{6}{4}  \right)
\end{align*}
The coefficients $1,4,1,16,9,\dots$ are given by the sequence [ID] in the OEIS~\cite{sloane2003line}.
To obtain the formula for double sum of powers, we simply apply summation operator over the ordinary sum again, thus
\begin{align*}
    \KnuthRFoldSum{2}{n}{m} = \sum_{j=0}^{m} \Delta^{j} t^{m} \left[ (-1)^{j} \sum_{k=1}^{n} \binom{j+t-1}{j+1} + \sum_{k=1}^{n} \binom{k-t+1}{j+1} \right]
\end{align*}
which yields
\begin{align*}
    \KnuthRFoldSum{2}{n}{m} = \sum_{j=0}^{m} \Delta^{j} t^{m} \left[ (-1)^{j} \binom{j+t-1}{j+1} n + \sum_{k=1}^{n} \binom{k-t+1}{j+1} \right]
\end{align*}
Thus,
\begin{proposition}[Double sums of powers via Newton's series]
    \begin{align*}
        \KnuthRFoldSum{2}{n}{m} = \sum_{j=0}^{m} \Delta^{j} t^{m} \left[ (-1)^{j} \binom{j+t-1}{j+1} n + (-1)^{j+1} \binom{j+t-1}{j+2} n^0 + \binom{n-t+2}{j+2} \right]
    \end{align*}
    \begin{proof}
        We have $\KnuthRFoldSum{2}{n}{m} = \sum_{j=0}^{m} \Delta^{j} t^{m} \left[ (-1)^{j} \binom{j+t-1}{j+1} n + \sum_{k=1}^{n} \binom{k-t+1}{j+1} \right]$,
        where $\sum_{k=1}^{n} \binom{k-t+1}{j+1} = (-1)^{j+1} \binom{j+t-1}{j+2} n^0 + \binom{n-t+2}{j+2}$ by means of segmented hockey stick
        identity~\eqref{prop:segmented-hockey-stick-identity}.
    \end{proof}
\end{proposition}
For example, given $t=5$, the double sums of powers are
\begin{align*}
    \KnuthRFoldSum{2}{n}{0} &= 1 \left( \binom{n-3}{2} + \binom{4}{1} n - \binom{4}{2} \right) \\
    \KnuthRFoldSum{2}{n}{1} &= 5 \left( \binom{n-3}{2} + \binom{4}{1} n - \binom{4}{2} \right) + 1 \left( \binom{n-3}{3} - \binom{5}{2} n + \binom{5}{3} \right) \\
    \KnuthRFoldSum{2}{n}{2} &= 25 \left( \binom{n-3}{2} + \binom{4}{1} n - \binom{4}{2} \right) + 11 \left( \binom{n-3}{3} - \binom{5}{2} n + \binom{5}{3} \right) \\
    &+ 2 \left( \binom{n-3}{4} + \binom{6}{3} n - \binom{6}{4} \right) \\
    \KnuthRFoldSum{2}{n}{3} &= 125 \left( \binom{n-3}{2} + \binom{4}{1} n - \binom{4}{2} \right) + 91 \left( \binom{n-3}{3} - \binom{5}{2} n + \binom{5}{3} \right) \\
    &+ 36 \left( \binom{n-3}{4} + \binom{6}{3} n - \binom{6}{4} \right) + 6 \left( \binom{n-3}{5} - \binom{7}{4} n + \binom{7}{5} \right)
\end{align*}
Similarly, we obtain the formula for the triple sums of powers
