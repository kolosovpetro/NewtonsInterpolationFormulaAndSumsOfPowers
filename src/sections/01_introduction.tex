In this manuscript we derive formulas for multifold sums of powers by utilizing Newton's interpolation formula.
Furthermore, we provide formulas for multifold sums of powers in terms of Stirling numbers of the second kind and Eulerian numbers.

Allow us to start from the definition of multifold sums of powers.
We utilize the recurrence proposed by Donald Knuth in his article
\textit{Johann Faulhaber and sums of powers}, see~\cite{knuth1993johann}
\begin{align*}
    \KnuthRFoldSum{0}{n}{m}   &= n^m \\
    \KnuthRFoldSum{1}{n}{m}   &= \KnuthRFoldSum{0}{1}{m} + \KnuthRFoldSum{0}{2}{m} + \cdots + \KnuthRFoldSum{0}{n}{m} \\
    \KnuthRFoldSum{r+1}{n}{m} &= \KnuthRFoldSum{r}{1}{m} + \KnuthRFoldSum{r}{2}{m} + \cdots + \KnuthRFoldSum{r}{n}{m}
\end{align*}
Throughout the paper, we utilize the Newton's interpolation formula as stated below
\begin{proposition}
    \label{proposition:newton-series-in-arbitrary-point-a}
    (Newton's series around arbitrary point~\cite[Lemma V]{newton1850newton}.)
    \begin{align*}
        f(x) = \sum_{j=0}^{\infty} \binom{x-a}{j} \Delta^{j} f(a)
    \end{align*}
    where $\Delta^{k} f(x) = \sum_{j=0}^{k} (-1)^{k-j} \binom{k}{j} f(x+j)$ is $k$-degree forward finite difference of $f$.
\end{proposition}
Which indeed holds, because
\begin{align*}
    n^3 &= 0 \binom{n}{0} + 1 \binom{n}{1} + 6 \binom{n}{2} + 6 \binom{n}{3} \\
    n^3 &= 1 \binom{n-1}{0} + 7 \binom{n-1}{1} + 12 \binom{n-1}{2} + 6 \binom{n-1}{3} \\
    n^3 &= 8 \binom{n-2}{0} + 19 \binom{n-2}{1} + 18 \binom{n-2}{2} + 6 \binom{n-2}{3}
\end{align*}

\begin{proposition}[Newton's series for power]
    For non-negative integers $m,n$ and an arbitrary integer $t$
    \label{prop:newton-series-for-monomial-reversed}
    \begin{align*}
        n^m = \sum_{k=0}^{m} \binom{n-t}{k} \Delta^{k} t^m
    \end{align*}
\end{proposition}

Thus, for an arbitrary integer $t$, the ordinary sum of powers is
\begin{align*}
    \KnuthRFoldSum{1}{n}{m}
    = \sum_{k=1}^{n} \sum_{j=0}^{m} \binom{-t+k}{j} \Delta^{j} t^m
    = \sum_{j=0}^{m} \Delta^{j} t^m  \sum_{k=1}^{n} \binom{-t+k}{j}
\end{align*}
\begin{proposition}[Segmented Hockey stick identity]
    \label{prop:segmented-hockey-stick-identity}
    For integers $n,t$ and $j$
    \begin{align*}
        \sum_{k=0}^{n} \binom{-t+k}{j} &= (-1)^j \binom{j+t}{j+1} +  \binom{n-t+1}{j+1}
    \end{align*}
\end{proposition}
Therefore,
\begin{proposition}[Ordinary sums of powers via Newton's series]
    \label{prop:ordinary-sums-of-powers-via-newtons-series}
    For non-negative integers $n,m$ and an arbitrary integer $t$
    \begin{align*}
        \KnuthRFoldSum{1}{n}{m} = \sum_{j=0}^{m} \Delta^{j} t^m \left[ (-1)^j \binom{j+t-1}{j+1} +  \binom{n-t+1}{j+1} \right]
    \end{align*}
    \begin{proof}
        Ordinary sum of powers is given by $\KnuthRFoldSum{1}{n}{m} = \sum_{j=0}^{m} \Delta^{j} t^m  \sum_{k=1}^{n} \binom{-t+k}{j}$,
        where $\sum_{k=1}^{n} \binom{-t+k}{j} =  (-1)^{j} \binom{j+t-1}{j+1} + \binom{n-t+1}{j+1}$
        by means of segmented hockey stick identity~\eqref{prop:segmented-hockey-stick-identity}.
    \end{proof}
\end{proposition}
The special cases for $t=0$ and $t=1$ are widely known and appear in literature quite frequently.
For $t=0$ and $m=3$ we have the famous identity
\begin{align*}
    \KnuthRFoldSum{1}{n}{3} = 0 \binom{n+1}{1} + 1 \binom{n+1}{2} + 6 \binom{n+1}{3} + 6 \binom{n+1}{4}
\end{align*}
which was discussed in~\cite[p. 190]{graham1994concrete} and in~\cite{pfaff2007deriving}.
The coefficients $0, 1, 6, 6, 0, 1, 14, 36, 24, \dots$ are given by the sequence
\href{https://oeis.org/A131689}{\texttt{A131689}} in the OEIS~\cite{sloane2003line}.

The special cases for $t=1$ and $m=2,3,4,5$ were discussed in~\cite{cereceda2022sums}.
For instance,
\begin{align*}
    \KnuthRFoldSum{1}{n}{3} &= 1 \binom{n}{1} + 7 \binom{n}{2} + 12 \binom{n}{3} + 6 \binom{n}{4} \\
    \KnuthRFoldSum{1}{n}{4} &= 1 \binom{n}{1} +15 \binom{n}{2} +50 \binom{n}{3} +60 \binom{n}{4} +24 \binom{n}{5}
\end{align*}
The coefficients $1,7,12,6,1,15, \dots$ are given by the sequence
\href{https://oeis.org/A028246}{\texttt{A028246}}
in the OEIS~\cite{sloane2003line}.
Interestingly enough that the paper~\cite{cereceda2022sums} gives the formula for sums of powers
\begin{align*}
    \KnuthRFoldSum{1}{n}{k} = \sum_{j=0}^{k} j! \left[ \binom{n+1-r}{j+1} + (-1)^j \binom{r+j-1}{j+1} \right] \stirlingii{k}{j}_{r}
\end{align*}
where $\stirlingii{k}{j}_{r}$ are generalized Stirling numbers of the second kind.
The formula above is identical to the proposition~\eqref{prop:ordinary-sums-of-powers-via-newtons-series},
which implies that finite differences can be expressed in terms of generalized Stirling numbers of the second kind,
that is $\Delta^{j} t^m = j! \stirlingii{m}{j}_{t}$.

By considering the special cases of the proposition~\eqref{prop:ordinary-sums-of-powers-via-newtons-series} for $t=4$,
we observe rather unexpected formulas for sums of powers, namely
\begin{align*}
    \KnuthRFoldSum{1}{n}{0} &= 1  \left( \binom{n-3}{1} + \binom{3}{1}  \right) \\
    \KnuthRFoldSum{1}{n}{1} &= 4  \left( \binom{n-3}{1} + \binom{3}{1}  \right)  + 1  \left( \binom{n-3}{2} - \binom{4}{2}  \right) \\
    \KnuthRFoldSum{1}{n}{2} &= 16 \left( \binom{n-3}{1} + \binom{3}{1}  \right)  + 9  \left( \binom{n-3}{2} - \binom{4}{2}  \right) + 2  \left( \binom{n-2}{3} + \binom{5}{3}  \right) \\
    \KnuthRFoldSum{1}{n}{3} &= 64 \left( \binom{n-3}{1} + \binom{3}{1}  \right)  + 61 \left( \binom{n-3}{2} - \binom{4}{2}  \right) + 30 \left( \binom{n-3}{3} + \binom{5}{3}  \right) \\ &+ 6 \left( \binom{n-3}{4} - \binom{6}{4}  \right)
\end{align*}
The coefficients $1,4,1,16,9,\dots$ are given by the sequence
\href{https://oeis.org/A391633}{\texttt{A391633}}
in the OEIS~\cite{sloane2003line}.
To obtain the formula for double sum of powers, we apply the summation operator over the ordinary sum of powers again,
thus
\begin{align*}
    \KnuthRFoldSum{2}{n}{m} = \sum_{j=0}^{m} \Delta^{j} t^{m} \left[ (-1)^{j} \sum_{k=1}^{n} \binom{j+t-1}{j+1} + \sum_{k=1}^{n} \binom{k-t+1}{j+1} \right]
\end{align*}
which yields
\begin{align*}
    \KnuthRFoldSum{2}{n}{m} = \sum_{j=0}^{m} \Delta^{j} t^{m} \left[ (-1)^{j} \binom{j+t-1}{j+1} n + \sum_{k=1}^{n} \binom{k-t+1}{j+1} \right]
\end{align*}
Thus,
\begin{proposition}[Double sums of powers via Newton's series]
    For non-negative integers $n,m$ and an arbitrary integer $t$
    \begin{align*}
        \KnuthRFoldSum{2}{n}{m} = \sum_{j=0}^{m} \Delta^{j} t^{m} \left[ (-1)^{j} \binom{j+t-1}{j+1} n + (-1)^{j+1} \binom{j+t-1}{j+2} n^0 + \binom{n-t+2}{j+2} \right]
    \end{align*}
    \begin{proof}
        We have $\KnuthRFoldSum{2}{n}{m} = \sum_{j=0}^{m} \Delta^{j} t^{m} \left[ (-1)^{j} \binom{j+t-1}{j+1} n + \sum_{k=1}^{n} \binom{k-t+1}{j+1} \right]$,
        where $\sum_{k=1}^{n} \binom{k-t+1}{j+1} = (-1)^{j+1} \binom{j+t-1}{j+2} n^0 + \binom{n-t+2}{j+2}$ by means of segmented hockey stick
        identity~\eqref{prop:segmented-hockey-stick-identity}.
    \end{proof}
\end{proposition}
For example, given $t=5$, the double sums of powers are
\begin{align*}
    \KnuthRFoldSum{2}{n}{0} &= 1 \left( \binom{n-3}{2} + \binom{4}{1} n - \binom{4}{2} \right) \\
    \KnuthRFoldSum{2}{n}{1} &= 5 \left( \binom{n-3}{2} + \binom{4}{1} n - \binom{4}{2} \right) + 1 \left( \binom{n-3}{3} - \binom{5}{2} n + \binom{5}{3} \right) \\
    \KnuthRFoldSum{2}{n}{2} &= 25 \left( \binom{n-3}{2} + \binom{4}{1} n - \binom{4}{2} \right) + 11 \left( \binom{n-3}{3} - \binom{5}{2} n + \binom{5}{3} \right) \\
    &+ 2 \left( \binom{n-3}{4} + \binom{6}{3} n - \binom{6}{4} \right) \\
    \KnuthRFoldSum{2}{n}{3} &= 125 \left( \binom{n-3}{2} + \binom{4}{1} n - \binom{4}{2} \right) + 91 \left( \binom{n-3}{3} - \binom{5}{2} n + \binom{5}{3} \right) \\
    &+ 36 \left( \binom{n-3}{4} + \binom{6}{3} n - \binom{6}{4} \right) + 6 \left( \binom{n-3}{5} - \binom{7}{4} n + \binom{7}{5} \right)
\end{align*}
The coefficients $1,5,1,25,11,2,\dots$ are given by the sequence
\href{https://oeis.org/A391635}{\texttt{A391635}}
in the OEIS~\cite{sloane2003line}.

Similarly, we obtain the formula for the triple sums of powers
\begin{proposition}[Triple sums of powers via Newton's series]
    For non-negative integers $n,m$ and an arbitrary integer $t$
    \begin{align*}
        \KnuthRFoldSum{3}{n}{m} &= \sum_{j=0}^{m} \Delta^{j} t^m \Bigg[ (-1)^{j} \binom{j+t-1}{j+1} \KnuthRFoldSum{2}{n}{0} + (-1)^{j+1} \binom{j+t-1}{j+2} \KnuthRFoldSum{1}{n}{0} + \\
        &+ (-1)^{j+2} \binom{j+t-1}{j+3} \KnuthRFoldSum{0}{n}{0} +  \binom{n-t+3}{j+3} \Bigg]
    \end{align*}
    \begin{proof}
        By summing up the double powers sums, we get
        \begin{align*}
            &\KnuthRFoldSum{3}{n}{m} = \sum_{j=0}^{m} \Delta^{j} t^m \sum_{k=1}^{n}  \left[ (-1)^{j} \binom{j+t-1}{j+1} k^1 + (-1)^{j+1} \binom{j+t-1}{j+2} k^0 + \binom{k-t+2}{j+2} \right] \\
            &= \sum_{j=0}^{m} \Delta^{j} t^m \left[ (-1)^{j} \binom{j+t-1}{j+1} \sum_{k=1}^{n} k^1 + (-1)^{j+1} \binom{j+t-1}{j+2} \sum_{k=1}^{n} k^0 + \sum_{k=1}^{n}  \binom{k-t+2}{j+2} \right]
        \end{align*}
        Note that $\sum_{k=1}^{n} k^1 = \KnuthRFoldSum{2}{n}{0}$ and $\sum_{k=1}^{n} k^0 = \KnuthRFoldSum{1}{n}{0}$.
        Thus,
        \begin{align*}
            \sum_{k=1}^{n}  \binom{k-t+2}{j+2} = (-1)^{j+2} \binom{j+t-1}{j+3} \KnuthRFoldSum{0}{n}{0} +  \binom{n-t+3}{j+3}
        \end{align*}
        by segmented hockey stick identity~\eqref{prop:segmented-hockey-stick-identity}.
        This completes the proof.
    \end{proof}
\end{proposition}
For example, given $t=4$, the triple sums of powers are
\begin{align*}
    \KnuthRFoldSum{3}{n}{0} &= 1 \left( \binom{n-1}{3} + \binom{3}{1} \KnuthRFoldSum{2}{n}{0} - \binom{3}{2} \KnuthRFoldSum{1}{n}{0} + \binom{3}{3} \KnuthRFoldSum{0}{n}{0} \right) \\
    \KnuthRFoldSum{3}{n}{1} &= 4 \left( \binom{n-1}{3} + \binom{3}{1} \KnuthRFoldSum{2}{n}{0} - \binom{3}{2} \KnuthRFoldSum{1}{n}{0} + \binom{3}{3} \KnuthRFoldSum{0}{n}{0} \right) \\
    &+ 1 \left( \binom{n-1}{4} - \binom{4}{2} \KnuthRFoldSum{2}{n}{0} + \binom{4}{3} \KnuthRFoldSum{1}{n}{0} - \binom{4}{4} \KnuthRFoldSum{0}{n}{0} \right) \\
    \KnuthRFoldSum{3}{n}{2} &= 16 \left( \binom{n-1}{3} + \binom{3}{1} \KnuthRFoldSum{2}{n}{0} - \binom{3}{2} \KnuthRFoldSum{1}{n}{0} + \binom{3}{3} \KnuthRFoldSum{0}{n}{0} \right) \\
    &+ 9 \left( \binom{n-1}{4} - \binom{4}{2} \KnuthRFoldSum{2}{n}{0} + \binom{4}{3} \KnuthRFoldSum{1}{n}{0} - \binom{4}{4} \KnuthRFoldSum{0}{n}{0} \right) \\
    &+ 2 \left( \binom{n-1}{5} + \binom{5}{3} \KnuthRFoldSum{2}{n}{0} - \binom{5}{4} \KnuthRFoldSum{1}{n}{0} + \binom{5}{5} \KnuthRFoldSum{0}{n}{0} \right)
\end{align*}

Continuing similarly, we are able to derive the formula for multifold sums of powers,
which is
\begin{theorem}[Multifold sums of powers via Newton's series]
    \label{theorem:multifold-sums-of-powers-via-newtons-series}
    \begin{mdframed}
        For non-negative integers $r,n,m$ and an arbitrary integer $t$
        \begin{align*}
            \KnuthRFoldSum{r}{n}{m} = \sum_{j=0}^{m} \Delta^{j} t^{m} \left[ \left( \sum_{s=1}^{r} (-1)^{j+s-1} \binom{j+t-1}{j+s} \KnuthRFoldSum{r-s}{n}{0} \right) + \binom{n-t+r}{j+r} \right]
        \end{align*}
    \end{mdframed}
    \begin{proof}
        By Newton's series for power~\eqref{prop:newton-series-for-monomial-reversed} and repeated
        applications of the segmented hockey stick identity~\eqref{prop:segmented-hockey-stick-identity}.
    \end{proof}
\end{theorem}
We may observe that
\begin{proposition}[Multifold sum of zero powers]
    \label{prop:multifold-sum-of-zero-powers}
    For integers $r$ and $n$
    \begin{align*}
        \KnuthRFoldSum{r}{n}{0} = \binom{r+n-1}{r}
    \end{align*}
    \begin{proof}
        By hockey stick identity $\sum_{k=0}^{t} \binom{j+k}{j} = \binom{j+t+1}{j+1}$.
    \end{proof}
\end{proposition}
Which yields the following binomial variations of the multifold
sums of powers~\eqref{theorem:multifold-sums-of-powers-via-newtons-series}
\begin{proposition}[Multifold sums of powers binomial form]
    For non-negative integers $r,n,m$ and an arbitrary integer $t$
    \begin{align*}
        \KnuthRFoldSum{r}{n}{m} = \sum_{j=0}^{m} \Delta^{j} t^{m} \left[ \left( \sum_{s=1}^{r} (-1)^{j+s-1} \binom{j+t-1}{j+s} \binom{r-s+n-1}{r-s} \right) + \binom{n-t+r}{j+r} \right]
    \end{align*}
\end{proposition}
\begin{proposition}[Multifold sums of powers binomial form reindexed]
    For non-negative integers $r,n,m$ and an arbitrary integer $t$
    \begin{align*}
        \KnuthRFoldSum{r}{n}{m} = \sum_{j=0}^{m} \Delta^{j} t^{m} \left[ \left( \sum_{s=0}^{r-1} (-1)^{j+s} \binom{j+t-1}{j+s+1} \binom{r-s+n-2}{r-s-1} \right) + \binom{n-t+r}{j+r} \right]
    \end{align*}
\end{proposition}
Finite differences of powers are closely related to Stirling numbers of the second kind
\begin{lemma}[Finite differences via Stirling numbers]
    \label{lem:finite-differences-via-stirling-numbers}
    For non-negative integers $j,m$ and an arbitrary integer $t$
    \begin{align*}
        \Delta^{j} t^{m} = \sum_{k=0}^{m} \binom{t}{k} \stirlingii{m}{j+k} (j+k)!
    \end{align*}
\end{lemma}
Which implies variations of the formulas for sums of powers
\begin{proposition}[Ordinary sums of powers via Stirling numbers]
    For non-negative integers $n,m$ and arbitrary integer $t$
    \begin{align*}
        \KnuthRFoldSum{1}{n}{m} = \sum_{j=0}^{m} \sum_{k=0}^{m} \left[ (-1)^j \binom{j+t-1}{j+1} +  \binom{n-t+1}{j+1} \right] \binom{t}{k} \stirlingii{m}{j+k} (j+k)!
    \end{align*}
    \begin{proof}
        By ordinary sums of powers via Newton's series~\eqref{prop:ordinary-sums-of-powers-via-newtons-series}
        and finite difference via Stirling numbers of the second kind~\eqref{lem:finite-differences-via-stirling-numbers}.
    \end{proof}
\end{proposition}
In general,
\begin{proposition}[Multifold sums of powers via Stirling numbers]
    For non-negative integers $r,n,m$ and an arbitrary integer $t$
    \begin{align*}
        \KnuthRFoldSum{r}{n}{m} = \sum_{j=0}^{m} \sum_{k=0}^{m} \left[ \left( \sum_{s=1}^{r} (-1)^{j+s-1} \binom{j+t-1}{j+s} \KnuthRFoldSum{r-s}{n}{0} \right) + \binom{n-t+r}{j+r} \right] \binom{t}{k} \stirlingii{m}{j+k} (j+k)!
    \end{align*}
    \begin{proof}
        By multifold sums of powers via Newton's series~\eqref{theorem:multifold-sums-of-powers-via-newtons-series}
        and finite difference via Stirling numbers of the second kind~\eqref{lem:finite-differences-via-stirling-numbers}.
    \end{proof}
\end{proposition}
The proposition above can be presented in a pure binomial form as well, by means of the identity~\eqref{prop:multifold-sum-of-zero-powers}: $\KnuthRFoldSum{r}{n}{0} = \binom{r+n-1}{r}$.

In addition, we are able to express multifold sums of powers via Eulerian numbers, by
expressing the forward finite difference via the Worpitzky identity~\cite{Worpitzky1883}
\begin{lemma}[Worpitzky identity]
    For non-negative integers $t,m$
    \begin{align*}
        t^{m} = \sum_{k=0}^{m} \eulerian{m}{k} \binom{t+k}{m}
    \end{align*}
\end{lemma}
where $\eulerian{n}{k}$ are Eulerian numbers.
Thus,
\begin{lemma}[Finite difference via Eulerian numbers]
    \label{lem:finite-differences-via-eulerian-numbers}
    For non-negative integers $j,m$ and an arbitrary integer $t$
    \begin{align*}
        \Delta^{j} t^{m} = \sum_{k=0}^{m} \eulerian{m}{k} \binom{t+k}{m-j}
    \end{align*}
\end{lemma}
Therefore,
\begin{proposition}[Multifold sums of powers via Eulerian numbers]
    For non-negative integers $r,n,m$ and an arbitrary integer $t$
    \begin{align*}
        \KnuthRFoldSum{r}{n}{m}= \sum_{j=0}^{m} \sum_{k=0}^{m} \left[ \left( \sum_{s=1}^{r} (-1)^{j+s-1} \binom{j+t-1}{j+s} \KnuthRFoldSum{r-s}{n}{0} \right) + \binom{n-t+r}{j+r} \right] \eulerian{m}{k} \binom{t+k}{m-j}
    \end{align*}
    \begin{proof}
        By multifold sums of powers via Newton's series~\eqref{theorem:multifold-sums-of-powers-via-newtons-series}
        and finite difference via Eulerian numbers of the second kind~\eqref{lem:finite-differences-via-eulerian-numbers}.
    \end{proof}
\end{proposition}
