In this manuscript we focus on the idea to combine the Newton's interpolation formula and Hockey-stick identity
for binomial coefficients to express the sums of powers seamlessly.

This particular idea is great, however it can be generalized even further, so that the main aim is to
utilize an interpolation formula for power $n^m$ in terms of \textit{abstract difference operator} $D(n^m)$ and binomial coefficients $\binom{f(n)}{k}$
such that $n$ indicates the variable of power function.
The difference operator can be arbitrary, for example: forward, backward, central differences etc.
For example, the abstract interpolation formula is
\begin{align*}
    n^m = \sum_{k} \binom{f(n)}{k} D(n^m, k)
\end{align*}
Thus, the formula of sums of powers involves the abstract difference operator $D$ in some point $k$ and hockey stick
identity over the binomial coefficients $\binom{n}{k}$
\begin{align*}
    \KnuthRFoldSum{1}{n}{m} = \sum_{k} D(n^m, k) \sum_{j \leq n} \binom{f(j)}{k}
\end{align*}
Similarly, for multifold sums of powers
\begin{align*}
    \KnuthRFoldSum{r}{n}{m} = \sum_{k} D(n^m, k) \binom{f(n+r)}{k}
\end{align*}
Many of interpolation approaches involve rising factorials $x_{(n)}$, falling factorials $(x)_n$ or usual factorials $n!$,
and thus can be expressed
in terms of binomial coefficients, because
\begin{align*}
    \frac{(x)_n}{n!} = \binom{x}{n}; \quad \frac{x_{(n)}}{n!} = \binom{x+n-1}{n}.
\end{align*}

In particular, Donald Knuth provides the formula multifold sums of odd powers~\cite{knuth1993johann}
such that based on the operator of central finite differences of power evaluated in zero, that is
\begin{proposition}[Multifold sums of odd powers]
    \label{prop:knuth-sums-of-odd-powers}
    \begin{align*}
        \KnuthRFoldSum{r}{n}{2m-1}
        &= \sum_{k=1}^{m} (2k-1)! T(2m, 2k) \binom{n+k-1+r}{2k-1+r} \\
        &= \sum_{k=1}^{m} \binom{n+k-1+r}{2k-1+r} \frac{1}{2k} \delta^{2k} 0^{2m}
    \end{align*}
\end{proposition}
where $T(n,k)$ are central factorial numbers of the second
kind, see~\cite[section 58]{steffensen1927interpolation} and~\cite[formula (10a)]{carlitz1963divided},
such that
\begin{align*}
    T(n, k) = \frac{1}{k!} \delta^k 0^n = \frac{1}{k!} \sum_{j=0}^{k} (-1)^j \binom{k}{j} \left( \frac{k}{2} - j \right)^n
\end{align*}
In general, the central factorial numbers of the second kind $T(n,k)$ were defined by Riordan in his
fundamental work \textit{Combinatorial identities}~\cite[ch. 6.5, formula (24)]{riordan1968combinatorial},
via the polynomial identity
\begin{lemma} [Riordan power identity]
    \begin{align*}
        n^m = \sum_{k=1}^{m} T(m,k) \centralFactorial{n}{k}
    \end{align*}
\end{lemma}
where $\centralFactorial{n}{k}$ are central factorials
$\centralFactorial{n}{k} = n \prod_{j=0}^{k-1} \left( n + \frac{k}{2} -j \right)$.
The sequence
\href{https://oeis.org/A008957}{\texttt{A008957}}
in the OEIS~\cite{sloane2003line} provides non-zero central factorial numbers of the second kind $T(2n,2k)$.

The Knuth's formula~\eqref{prop:knuth-sums-of-odd-powers} utilizes the operator
of central finite differences of power evaluated in zero,
it is worth to research the existence of the sums of odd powers involving the central differences
evaluated in arbitrary integer point $t$,
similar to multifold sums of powers via Newton's series~\eqref{theorem:multifold-sums-of-powers-via-newtons-series}.
